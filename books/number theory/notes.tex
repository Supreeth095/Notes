\documentclass [11pt]{article}
\usepackage{amssymb}
\title{Introduction to Number Theory}
\author{Bui Hong Ha}
\date{\today}

\begin{document}
\maketitle
\begin{abstract}
  Introduction to Number Theory's note. On this note, I summerize many main ideas, important definition, and critical exercises when reading the book: ``introduction to number theory `` of professor Victor Shoup.
\addcontentsline{toc}{section}{abstract}
\end{abstract}

\tableofcontents

\section{Premilinaries}
Some terminology, notation, and simple facts that will bed used throughout the text\\
\begin{center}
 \textbf{ Logarithms and exponentials}
\end{center}
We write $\log x$ for the natural logarithm, and $\log_b x $ for the logarithm of x to the base b.\\
We write $e^x$ for the usual exponential function, where $e \approx 2.71828 $ is the base of the natural logarithm. $ exp[x] \approx e^x $

\begin{center}
  \textbf{Sets and families}
\end{center}
We use standard set-theoretic notation: \\
\begin{itemize} 
\item $\emptyset$ denotes the empty set
\item $x \in A$ means that x is an element, or member, of the set A 
\item For two sets $A, B, A \subset B $ means that $A$ is a subset of $B$ (with $A$ possibly equal to $B$), and $A \subsetneq B$ means that $A$ is a proper subset of $B$
\item $A \cup B$ denotes the union of $A$ and $B$
\item $A \cap B$ denotes the intersection of $A$ and $B$
\item $A / B$ denotes the set of all elements of $A$ that are not in $B$
\item if A is a set with a finite number of elements, then we write |A| for its \textbf{size}, or \textbf{cardinality}
\item $S_1 \times \ldots \times S_n$ for the \textbf{Cartesian product} of sets $S_1 ,\ldots,S_n$, which is the set of all \textit{n}-tuples $(a_1, \ldots, a_n)$, where $a_i \in S_i$ for $i = 1,\ldots,n$. We write $S^{\times n}$ for the Cartesian product of n copies of a set S, and for $x \in S$, we write $x^{\times n}$ for the element of $S^{\times n}$ consisting of n copies of x.
\item A \textbf{family} is a collection of objects, indexed by some set \textit{I}, called an \textbf{index set}. If for each $i \in I$ we have an associated object $x_i$, the family of all such objects is denoted by $\{x_i\}_{i \in I}$. Unlike a set, a family may contain duplicates,: that is we may have $x_i = x_j$ for some pair of indices $i,j$ with $i \neq j$. 
\end{itemize}

\begin{center}
\textbf{Functions}
\end{center}
\begin{itemize}
\item $f \colon A \rightarrow B $ indicates that $f$ is a function (also called a \textbf{map}) from a set $A$ to a set $B$
\item If $A' \subseteq A$, then $f(A') := \{ f(a) : a \in A' \}$ is the \textbf{image} of $A'$ undef $f$, and $f(A)$ is simply referred to as the \textbf{image} of $f$; if $B' \subseteq B$, then $f^{-1}(B'):= \{ a \in A: f(a) \in B' \}$ is the \textbf{pre-image} of $B'$ under $f$
\end{itemize}

\section{Ideals and greatest common divisors}
\end{document}
